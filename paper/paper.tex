% This is "sig-alternate.tex" V2.0 May 2012
% This file should be compiled with V2.5 of "sig-alternate.cls" May 2012
%
% This example file demonstrates the use of the 'sig-alternate.cls'
% V2.5 LaTeX2e document class file. It is for those submitting
% articles to ACM Conference Proceedings WHO DO NOT WISH TO
% STRICTLY ADHERE TO THE SIGS (PUBS-BOARD-ENDORSED) STYLE.
% The 'sig-alternate.cls' file will produce a similar-looking,
% albeit, 'tighter' paper resulting in, invariably, fewer pages.
%
% ----------------------------------------------------------------------------------------------------------------
% This .tex file (and associated .cls V2.5) produces:
%       1) The Permission Statement
%       2) The Conference (location) Info information
%       3) The Copyright Line with ACM data
%       4) NO page numbers
%
% as against the acm_proc_article-sp.cls file which
% DOES NOT produce 1) thru' 3) above.
%
% Using 'sig-alternate.cls' you have control, however, from within
% the source .tex file, over both the CopyrightYear
% (defaulted to 200X) and the ACM Copyright Data
% (defaulted to X-XXXXX-XX-X/XX/XX).
% e.g.
% \CopyrightYear{2007} will cause 2007 to appear in the copyright line.
% \crdata{0-12345-67-8/90/12} will cause 0-12345-67-8/90/12 to appear in the copyright line.
%
% ---------------------------------------------------------------------------------------------------------------
% This .tex source is an example which *does* use
% the .bib file (from which the .bbl file % is produced).
% REMEMBER HOWEVER: After having produced the .bbl file,
% and prior to final submission, you *NEED* to 'insert'
% your .bbl file into your source .tex file so as to provide
% ONE 'self-contained' source file.
%
% ================= IF YOU HAVE QUESTIONS =======================
% Questions regarding the SIGS styles, SIGS policies and
% procedures, Conferences etc. should be sent to
% Adrienne Griscti (griscti@acm.org)
%
% Technical questions _only_ to
% Gerald Murray (murray@hq.acm.org)
% ===============================================================
%
% For tracking purposes - this is V2.0 - May 2012

\documentclass{acm_proc_article-sp}

\begin{document}

\title{Machine Learning with League of Legends}

\numberofauthors{1} %  in this sample file, there are a *total*
% of EIGHT authors. SIX appear on the 'first-page' (for formatting
% reasons) and the remaining two appear in the \additionalauthors section.
%
\author{
% You can go ahead and credit any number of authors here,
% e.g. one 'row of three' or two rows (consisting of one row of three
% and a second row of one, two or three).
%
% The command \alignauthor (no curly braces needed) should
% precede each author name, affiliation/snail-mail address and
% e-mail address. Additionally, tag each line of
% affiliation/address with \affaddr, and tag the
% e-mail address with \email.
%
% 1st. author
\alignauthor
       Zezhou Liu\qquad Richard Cho \\
       ~\\
%         \vspace{10 pt}
        {\fontsize{13 pt}{1 em}\sffamily \selectfont Harvard University }\\
%         \showfont\\ 
%         {\fontsize{13 pt}{1 em} \selectfont \showfont }\\
%         {\fontsize{13 pt}{1 em}\sffamily \selectfont \showfont }\\
       \vspace{3 pt}
       {\fontsize{10 pt}{1 em}\sffamily \selectfont \{zezhouliu, rcho\}@college.harvard.edu}\\
}
% There's nothing stopping you putting the seventh, eighth, etc.
% author on the opening page (as the 'third row') but we ask,
% for aesthetic reasons that you place these 'additional authors'
% in the \additional authors block, viz.
% \additionalauthors{Additional authors: John Smith (The Th{\o}rv{\"a}ld Group,
% email: {\texttt{jsmith@affiliation.org}}) and Julius P.~Kumquat
% (The Kumquat Consortium, email: {\texttt{jpkumquat@consortium.net}}).}
% \date{30 July 1999}
% Just remember to make sure that the TOTAL number of authors
% is the number that will appear on the first page PLUS the
% number that will appear in the \additionalauthors section.

\graphicspath{{figures/}}

\maketitle

\begin{abstract}
The online electronic sports (eSports) community has grown significantly 
over the past decade, with over 1.7B people who play video games by 2014 
and 2B by 2015. 
Gamers spend more than 3 billion hours per week playing 
video games, generating over 400 Terabytes of data per week.  
This number is expected to increase dramatically as the growth 
rate of the eSports community continue to rise.  The potential applications 
for this deluge of data can be drawn from the parallels to traditional sports.

Data analysis has been used in the traditional sports settings to predict 
useful statistics about sports games even before the games occur.  
Machine learning models benefit from the large amounts of data available 
in past sports games and data anlysists can use a variety of tools to 
predict the winning team, the end-game score, and even 
individual player performances of future games.  

Machine learning is a natural application for the emerging eSports field 
given the large amounts of data available and the promise to mimic the 
traditional sports setting.  We show that by applying several machine learning 
techniques on the data available for League of Legends, we can accurately 
extract meaningful statistics such as win-rates.  
\end{abstract}



% A category with the (minimum) three required fields
%\category{H.4}{Information Systems Applications}{Miscellaneous}
%A category including the fourth, optional field follows...
%\category{D.2.8}{Software Engineering}{Metrics}[complexity measures, performance measures]

%\terms{Theory}

%\keywords{ACM proceedings, \LaTeX, text tagging}

\section{Introduction}


% The online electronic sports (eSports) community has grown significantly 
% over the past decade, with over 1.7B people who play video games in 2014.  
% This number comparable 
% to the number of people who actively participate in at least one sport (1.6B).  
% On a global scale, the number of esports enthusiasts compares well to mid-tier 
% traditional sports such as swimming or ice hockey.  By 2017, the number of esports 
% fans will come approach that of American football.  

% The market for competitive gaming now exceeds \$600 
% million, with a projected revenue growth rate of +29\% for 2015 and audience 
% growth rate of +20\%.  

\textbf{ Contributions.} Our contributions are summarized as follows:
\begin{itemize}
\item 1
\item 2
\item 3
\end{itemize} 

\section{Background and Related Work}

As mentioned in an earlier section, machine learning has numerous applications
throughout various fields.  Our line of work draws inspiration from 
several areas of research, including the classification of game outcomes and 
prediction of game statistics.  

\textbf{Machine Learning. }
Machine learning is an area in computer science that studies the construction of 
algorithms that can learn from and make predictions on data \cite{5_wikipedia_2015}.  
Relevant machine learning techniques include Principal Component Analysis (PCA), 
sparse coding, support vector machine, and random forests.  PCA is a technique 
commonly used for dimension reduction, which helps with identifying better features 
for a classification or regression task. Sparse coding is a way to automatically 
create a sparse representation of a given input using a set of identified feature 
patterns.  Support vector machines are supervised learning models that analyze and 
recognize patterns, and they can be used as a linear classifier.  Random forests 
are an ensemble learning method for classification that leverages the idea that 
individual classifiers may not be accurate but a group of them may be accurate and 
overcome overfitting.  Our work uses these machine learning techniques to create a 
model to predict the winning team in a League of Legends match.

\textbf{Applications in Games. }
Research by Nicolo Cesa-Bianchi and Gabor Lugosi discuss techniques to aid in 
prediction of games.  Their work concerns prediction with regards to both 
short-term and long-term forecasting.  In chess, the former might be used for 
planning a couple turns 
ahead, whereas the latter may be used for planning a winning strategy.  Their work 
combines previous work on statistical decision theory, game theory, and machine learning 
amongst other related areas.  We employ some of these motivating ideas in developing our
machine learning model.  

\textbf{PageRank. }The algorithm of PageRank emerged from the work of Sergey Brin and Larry 
Page while developing an algorithm to assign values and rankings to web pages \cite{3_page_brin_1998}.  
PageRank highlights several techniques to distinguish useful information 
from junk.  It also details a way to quantify the relationships between different 
useful information.  Since a League of Legends match consists of a team of five versus 
another team of five, the players on each team and their past match histories should 
affect the prediction of which team will win.  

\textbf{LoL Statistics. }
Most League of Legends services provide statistics that are intended to inform about certain statistics about individual champions such as the champion's win rates, play-rate, etc.  There has been little to no prior work in applying machine learning to create models for prediction or classification tasks.

\begin{figure}[t!]
  \centering
    \includegraphics[width=0.5\textwidth]{basic-statistics}
  \caption{Data analysis on games have so far focused on identifying individual statistics, rather than creating models for predicting statistics \cite{6_championgg_2015}. }
  \label{fig:basic-stats}
\end{figure}


\section{Motivation}
The increasing success and applicability of machine learning models in traditional 
sports inspires us to replicate its potential in competitive gaming.  
The competitive scene for League of Legends has been expanding aggressively, 
with many teams hiring analysts to gain deeper insights in the game.  We strongly 
believe that machine learning can play a large role in discovering hidden insights beyond basic stats.  
\section{Approach and Secret Weapon  }

\textbf{Naive Post-match. }

Our naive approach analyzed post-match data and tried to analyze whether a 
naive classifier could obtain a higher accuracy given all the information.  
Post-match data contains all the information about the match (except the winning 
team was withheld), including the number of kills on each team, the amount of minions 
that each team kind, the accumulation of gold for each player, etc.  We first pre-process 
the data to create a flattened vector representing the match.  
Given a set of training matches, we can apply the K-Clustering algorithm to train a 
set of \"features\" based on these training matches.  Using Orthogonal Matching Pursuit, we can then create a sparse code representation for these training matches.  
We can then train an SVM classifier using these sparse codes and the 
match outcomes to use for predicting the outcomes of future matches.  
We can test the SVM model by repeating the same process with a new testing data set.  Given that this approach has access to all the post-match data, 
we expect some variables to be highly correlated with the match 
outcome (more kills or more gold probably implies victory).

\textbf{Pre-match. }

Our next approach attempts to analyze only the pre-match data in order to create 
a classifier for the match's outcome.  In this case, we only examine variables 
intrinsic to the game itself rather than the performance of the teams and members 
during the match.  An example of an in-game intrinsic is if the team that has a 
certain champion (or set of champions) always seems 
to win or lose.  In this case, we would ignore the statistics of the matches 
actual performance, and instead try to discover whether there are certain inherent 
characteristics of the game that lead to certain advantages.  Our work-flow is similar 
to the one used in the post-game technique.

\textbf{Secret Weapon: Match Histories.  }
Our goal was to be able to predict which team would win a certain match before it occurs. 
We could therefore leverage the knowledge of matches in the past, as well as the pre-game 
information for this certain game.  Our naive post-match method is too constricting, 
requiring us to know the entire post-match data in order to classify the winner.  Our pre-match 
method is not optimal, since it forgoes all the important information that can be extracted 
from past match histories.  Match histories can generate useful information such as 
overall skill level, trends in some of the previously mentioned features, 
affinities for particular champions, and most importantly relationships with other 
players.  Match histories give deeper insight between the possible synergy between 
players that are matched together.  

Analyzing the recent match histories for each of the players in the game 
(five on each team, ten total) would be helpful to build a sophisticated model for the predictor.  
We limit match history calculations to the most recent 150 matches in order to 
minimize data explosion.  Even with this 
limitation, incorporating match histories significantly inflates the amount of 
data required and processed to analyze a single match.  
For example, to process a single match, we now require data from $1 \mbox{ match} * 10 \frac{\mbox{players}}{\mbox{match}} * 150 \frac{\mbox{games}}{\mbox{player}} = 1500 \mbox{ games}$ to be analyzed. 

\textbf{Data Wrangling. }

Match histories are most useful when you can determine relationships between the players. 
Single occurences of each relationship would be of little help (i.e., Player 1 and 2 show up on the same or different team only once).
Moreover, the large data explosion that occurs when gathering match history data makes this scenario even less desirable.  However, by carefully selecting the proper subset of matches to analyze, we can increase the overlap of players between matches. League of Legends has a division system (divisions which are called Bronze, Silver, Gold, Platinum, Diamond, Master, and Challenger) which breaks down players based on an ELO-type of rating. The trend seems to show that the higher the division, the less players there are in that division. 

%\includegraphics[scale=.5]{figures/rankedbreakdown}

Due to this system, people in the higher brackets are matched with the same people much more frequently. We can thus increase the effectiveness of our computation by selecting games of Challenger players. This was seen as in our analysis of 543 games, we archived the history for 787 unique players. This ratio is $\frac{787}{543} = 1.44$ players per game, which is far better than the ratio of 10 unique players per game which we would expect if we looked at a completely random match. 

After acquiring the player histories, the data needs to be pre-processed and converted into a format that can be inputted into an ML model.  As with our previous convention, we want to create a vector (or a matrix in this case, since we have 10 players and multiple features) to represent each match. 

Although applying PCA would be ideal for reducing the dimensions of the data while maintaining variance, we explicitly chose a relatively dense set of features based on 
specific domain knowledge of the game.  The chosen set of features are (\"Kills\", \"Deaths\", \"Assists\", \"GoldEarned\", \"DamageToChampions\", \"DamageOverall\", \"GamesPlayed\", \"Wins\"), which should be mostly uncorrelated. This is a good introductory feature set, as though it lacks some of the more complex to express features which mirror complex strategies in the game, it contains the basics.  This step replaces the need to use K-Means clustering to train a feature set. 

Once these features were extracted and transformed into a CSV format easily consumable by the sklearn python module, the data is randomly split into a training and testing set and fed it into a SVM model and a Random Forest Classifier. In this case, no sparsifying of the dataset was done before-hand, as the set of features is expected to be dense.  

\section{System Description. }

The system for the first two techniques has a straight pipeline starting with data pre-processing, to feature extraction, sparse coding, SVM training, and then classification.  This process is described in at a high level in the previous section under Naive Post-match and Pre-match.  This entire process is implemented in Python, using the sklearn library for machine learning.  The sklearn library contains all of the required functionality to do K-Means clustering and SVM classifying.  Many of the components were parallelized using OpenMP for Python.  


\includegraphics[scale=.6]{systemdescription}

We focus our system description on the workflow that models over match histories.  
The first stages (Steps A,B,C) of this process was to extract features from the data and convert this data into a consumable format. This process is described at a high level in the previous section under Data Wrangling.

The data collection stage is done with a Java application using the Orianna project, which wraps the Riot API with a robust interface that among many things allows the retrieved data to be serialized to disk. This step can further be divided into three parts:

\begin{enumerate}
\item Given a player, search through his match history. For every match found iterate through all 10 players and download the match history of each player. This step is very time and disk space expensive, as many many thousands of matches but be written to disk. Also, due to the fact that the Riot API key is rate limited, parallelism is of little use here as the rate limit is the main bottleneck. (This is step A)
\item Given a match, we have 10 players. For each player, iterate through the 150 game history. For each player, create a feature set. In this specific case, we aggregated (summed) certain numeric stats for each player. Align these feature sets so that they are in the form TEAM A:TEAM B:DID TEAM A WIN, where TEAM A = PLAYER 1 STATS:PLAYER 2 STATS: ... : PLAYER 5 STATS and likewise for TEAM B. This is then an array of longs, which is saved to disk. This step is time consuming; in the current Java version multithreading is implemented to accelerate the speed at which matches can be processed. (This is step B)
\item Given a set of matches, convert these into a CSV. This is by far the easiest step, no more than converting a group of serialized long arrays into one CSV. (This is step C)

\end{enumerate}

The second part of the previously detailed process is implemented using Hadoop MapReduce in Java (Step B:H). This allows for much faster processing of matches in parallel. However, it is very memory intensive (thousands of matches must be loaded into memory for analysis) so depending of hardware constraints 
the Java multi-threaded version could be faster/more optimal to use. 

The second stage (Step D) of this project is to once data has been aggregated, to consume this data and produce a model, and to see how accurate this model is. As stated earlier, sklearn was used, specifically SVM and Random Decision Forests. This train and test models using cross validation if opted for, and to save the model to disk if opted for. 


\section{Performance Evaluation}
\textbf{Speed}

In all of the workflows above, the data collection and pre-processing took the bulk of the time.  We focus on the description of our main sophisticated system, but the results are very similar in our first two workflows.  Our data collection is limited by the API-rate allowed by the developers of League of Legends.  As of now, there is no real way to get around the API limit without obtaining a 
production key or exploiting multiple API keys. Step 1 took over 100 compute hours to aggregate the 512 datapoints into the final data format.

\textbf{Accuracy}

\includegraphics[scale=.5]{modelscore}

The model, using cross validation, hovers at around 90\% accuracy over 384 test and 128 train datapoints. (Randomly segregated into those splits given an initial 512-sample set) This is surprisingly high for my expectations.
\section{Conclusion}
Our machine learning model evolved continuously as the project developed.  Each model gave us further insights into the next steps, eventually leading to a model that attempts to incorporate as much information as available in the data.  The results looks very promising in the current stage, with over 90\% accuracy in predicting match outcomes. However, the computation remains very expensive.  Aggregating the data to train on a single match requires collecting 1500 matches the ten players' recent histories. Future work will be done on trying to optimize this further so that possibly this may be deployable as a service. Acquiring a production key with a much higher rate-limit will also allow for faster data collection. The high-levels of accuracy may be due to overfitting since we have a relatively small sample set. The likelihood of overfitting should shrink as we continue to collect larger data sets.

\bibliographystyle{abbrv}
\bibliography{paper}
\nocite{*}

%\balancecolumns
%\balancecolumns % GM June 2007
% That's all folks!
\end{document}
