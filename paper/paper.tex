% This is "sig-alternate.tex" V2.0 May 2012
% This file should be compiled with V2.5 of "sig-alternate.cls" May 2012
%
% This example file demonstrates the use of the 'sig-alternate.cls'
% V2.5 LaTeX2e document class file. It is for those submitting
% articles to ACM Conference Proceedings WHO DO NOT WISH TO
% STRICTLY ADHERE TO THE SIGS (PUBS-BOARD-ENDORSED) STYLE.
% The 'sig-alternate.cls' file will produce a similar-looking,
% albeit, 'tighter' paper resulting in, invariably, fewer pages.
%
% ----------------------------------------------------------------------------------------------------------------
% This .tex file (and associated .cls V2.5) produces:
%       1) The Permission Statement
%       2) The Conference (location) Info information
%       3) The Copyright Line with ACM data
%       4) NO page numbers
%
% as against the acm_proc_article-sp.cls file which
% DOES NOT produce 1) thru' 3) above.
%
% Using 'sig-alternate.cls' you have control, however, from within
% the source .tex file, over both the CopyrightYear
% (defaulted to 200X) and the ACM Copyright Data
% (defaulted to X-XXXXX-XX-X/XX/XX).
% e.g.
% \CopyrightYear{2007} will cause 2007 to appear in the copyright line.
% \crdata{0-12345-67-8/90/12} will cause 0-12345-67-8/90/12 to appear in the copyright line.
%
% ---------------------------------------------------------------------------------------------------------------
% This .tex source is an example which *does* use
% the .bib file (from which the .bbl file % is produced).
% REMEMBER HOWEVER: After having produced the .bbl file,
% and prior to final submission, you *NEED* to 'insert'
% your .bbl file into your source .tex file so as to provide
% ONE 'self-contained' source file.
%
% ================= IF YOU HAVE QUESTIONS =======================
% Questions regarding the SIGS styles, SIGS policies and
% procedures, Conferences etc. should be sent to
% Adrienne Griscti (griscti@acm.org)
%
% Technical questions _only_ to
% Gerald Murray (murray@hq.acm.org)
% ===============================================================
%
% For tracking purposes - this is V2.0 - May 2012

\documentclass{acm_proc_article-sp}

\begin{document}

\title{Machine Learning with League of Legends}

\numberofauthors{1} %  in this sample file, there are a *total*
% of EIGHT authors. SIX appear on the 'first-page' (for formatting
% reasons) and the remaining two appear in the \additionalauthors section.
%
\author{
% You can go ahead and credit any number of authors here,
% e.g. one 'row of three' or two rows (consisting of one row of three
% and a second row of one, two or three).
%
% The command \alignauthor (no curly braces needed) should
% precede each author name, affiliation/snail-mail address and
% e-mail address. Additionally, tag each line of
% affiliation/address with \affaddr, and tag the
% e-mail address with \email.
%
% 1st. author
\alignauthor
       Zezhou Liu\qquad Richard Cho \\
       ~\\
%         \vspace{10 pt}
        {\fontsize{13 pt}{1 em}\sffamily \selectfont Harvard University }\\
%         \showfont\\ 
%         {\fontsize{13 pt}{1 em} \selectfont \showfont }\\
%         {\fontsize{13 pt}{1 em}\sffamily \selectfont \showfont }\\
       \vspace{3 pt}
       {\fontsize{10 pt}{1 em}\sffamily \selectfont \{zezhouliu, rcho\}@college.harvard.edu}\\
}
% There's nothing stopping you putting the seventh, eighth, etc.
% author on the opening page (as the 'third row') but we ask,
% for aesthetic reasons that you place these 'additional authors'
% in the \additional authors block, viz.
% \additionalauthors{Additional authors: John Smith (The Th{\o}rv{\"a}ld Group,
% email: {\texttt{jsmith@affiliation.org}}) and Julius P.~Kumquat
% (The Kumquat Consortium, email: {\texttt{jpkumquat@consortium.net}}).}
% \date{30 July 1999}
% Just remember to make sure that the TOTAL number of authors
% is the number that will appear on the first page PLUS the
% number that will appear in the \additionalauthors section.

\graphicspath{{figures/}}

\maketitle

\begin{abstract}

\end{abstract}



% A category with the (minimum) three required fields
%\category{H.4}{Information Systems Applications}{Miscellaneous}
%A category including the fourth, optional field follows...
%\category{D.2.8}{Software Engineering}{Metrics}[complexity measures, performance measures]

%\terms{Theory}

%\keywords{ACM proceedings, \LaTeX, text tagging}

\section{Introduction}

Machine Learning and Big Data have been hot, related topics in Computer Science.
Machine learning models benefit greatly from large amounts of data 
available for training and testing, and their applications are being used in varying fields 
of research and industry.  One of the most common use cases is to train a classifier, which 
can classify an input as a certain category.

Machine learning has been prevalent in the traditional sports setting for many different use 
cases ranging from something as general as predicting the winner of a game, to predicting something 
specific such as the score of a game, and to even predicting all the outcomes of the games in a 
tournament bracket.  Examples of the latter include the sophisticated models by Bloomberg to predict 
the World Cup with 16 teams and 32 games in a single-elimination tournament.  

With the rise of popularity of electronic sports (eSports) and competitive gaming, 
there is potential to draw a parallel in applications between eSports and traditional sports.  
In particular, League of Legends, an online multiplayer game, is one of the top contenders for 
the most professional and developed competitive eSport.  Professional teams for League of Legends 
operate similar to traditional sports teams, and they have begun hiring coaches and analysts to 
help them improve their gameplay.
These games generate tons of data that can be analyzed in a way 
similar to the results of sports games, potentially giving an advantage to the teams 
with better data analysis. 

Our work applies some 
machine learning techniques on the data available for League of Legends, and we show that
we can extract useful insights to predict win-rates.

\textbf{ Contributions.} Our contributions are summarized as follows:
\begin{itemize}
\item We apply machine learning techniques such as Sparse Coding, Support Vector Machines (SVM), 
and Random Forests on accumulated League of Legends 
game data to accurately predict win-rates for games. 
\item We introduce the idea of using a player's match history as an input to machine learning 
models as a training feature.
\item We demonstrate a framework to collect and aggregate data to feed into our machine learning model.
\end{itemize} 

\section{Background and Related Work}

\section{Adaptive Denormalization}
\section{Conclusion}
Our machine learning model evolved continuously as the project developed.  Each model gave us further insights into the next steps, eventually leading to a model that attempts to incorporate as much information as available in the data.  The results looks very promising in the current stage, with over 90\% accuracy in predicting match outcomes. However, the computation remains very expensive.  Aggregating the data to train on a single match requires collecting 1500 matches the ten players' recent histories. Future work will be done on trying to optimize this further so that possibly this may be deployable as a service. Acquiring a production key with a much higher rate-limit will also allow for faster data collection. The high-levels of accuracy may be due to overfitting since we have a relatively small sample set. The likelihood of overfitting should shrink as we continue to collect larger data sets.

\bibliographystyle{abbrv}
\bibliography{paper}
\nocite{*}

%\balancecolumns
%\balancecolumns % GM June 2007
% That's all folks!
\end{document}
